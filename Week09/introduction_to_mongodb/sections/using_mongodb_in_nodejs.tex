\section{Using MongoDB with NodeJS}\label{sec:using_mongodb_with_nodejs}
In this section we give a simple example how we can connect to a MongoDB database from NodeJS. For that we use the Node.js MongoDB Driver\footnote{\url{http://docs.mongodb.org/ecosystem/drivers/node-js/}}.

First we need to install the drive with npm as follows.

\begin{lstlisting}
npm install mongodb --save
\end{lstlisting}

Next, we write a function for fetching all the todos which we added previously. Let's place that into a file called \texttt{todos.js} and add the following code.

\begin{lstlisting}
'use strict';

const mongodb = require('mongodb');
const MongoClient = mongodb.MongoClient;
const mongoUrl = 'mongodb://localhost:27017/myapp';

const getTodo = function(filter, cb) {

  MongoClient.connect(mongoUrl, function(err, db) {
    if (err) {
      cb(err);
      db.close();
      return;
    }

    const collection = db.collection('todo');

    collection.find(filter, function(err, docs) {
      docs.toArray(function(err, d) {
        cb(err, d);
        db.close();
      });
    });
  });
};

getTodo({}, function(err, docs) {
  console.log(docs);
});
\end{lstlisting}

This code will connect to your local MongoDB and fetch all the added todos.
