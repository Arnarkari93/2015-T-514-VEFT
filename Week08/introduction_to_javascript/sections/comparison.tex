\section{Comparison}\label{sec:comparison}
In this section we discuss comparison operators that can be used to compare objects in JavaScript.


\subsection{Equality operators}
There are two types of Equality operators called Equality, and strict equality. They are \texttt{==} and \texttt{===} respectively.

The equality operator converts the operands if they are not of the same type. Here is an example.

\begin{lstlisting}
> 1 == '1'
true
\end{lstlisting}

The \texttt{number} operator is casted to string for the equality check. When we use this equality operator the operands do need to be of the same type.

The strict equality on the other hand will not evaluate anything to \texttt{true} unless they are of the same type.

\begin{lstlisting}
> 1 === '1'
false
\end{lstlisting}

Note that when you compare mutable objects, such as Arrays and Objects your a comparing the pointer to the object. They will not evaluate to \texttt{true} even if they are the same. Only if their pointers are the same. When you have immutable objects, such as numbers and booleans, their value are compared. 

When in doubt which one to use, always use strict equality. In matter of fact, always use strict equality unless you know what you are doing.

The negation equality check works the same, there we have \texttt{!=} and \texttt{!==}. The follow the same rules as we described above.

\subsection{Relational operators}
JavaScript supports the same relational operators as in C or Java. They are explained with examples in the following listing.

\begin{lstlisting}
> 2 > 1
true
> 3 <= 3
true
> 3 <= 3 && 2 != 2
false
\end{lstlisting}

