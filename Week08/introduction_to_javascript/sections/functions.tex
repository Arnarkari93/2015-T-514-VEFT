\section{Functions}\label{sec:functions}
In this section we introduce functions in JavaScript. Here we cover two types of function types, namely the functions defined with the \texttt{function} keyword and Fat Arrow Functions.

\subsection{Defining functions with the \texttt{function} keyword}
Since ES6, JavaScript now provides two ways to implement functions. The one prior ES6 uses the keyword \texttt{function}. Functions using that keyword are defined as follows.

\begin{lstlisting}
function add(x, y) {
    return x + y
}

add(1, 4)
\end{lstlisting}

Here we define a named function, with the name add which takes two arguments, \texttt{x} and \texttt{y}, and returns the sum of them. If a function does not have a return statement, then \texttt{undefined} is returned.

As stated above, functions are also objects and can be bound to variables and passed into other functions as arguments. This phenomenon is often called Higher Order Functions\footnote{\url{https://en.wikipedia.org/wiki/Higher-order_function}}.

\begin{lstlisting}
> const d = function(x, y) {return x + y};
undefined
> d(2, 5)
7
> const b = d;
undefined
> b(2, 5)
7
>
\end{lstlisting}

Here we create Anonymous function\footnote{\url{https://en.wikipedia.org/wiki/Anonymous_function}} and we assign it to a variable with the name \texttt{d}. Then later, we assign the same function to a new variable and use that variable to call the function.

Here is a simple example how we can pass methods as parameters into functions.

\begin{lstlisting}
> let sayHello = function(name) { 
   return "Hello " + name
  }
undefined
> let sayGoodbye = function(name) {
   return "See you later " + name
  }
undefined
> function greet(name, f) { return f(name)}
undefined
> greet('hlysig', sayHello)
'Hello hlysig'
> greet('hlysig', sayGoodbye)
'See you later hlysig'
\end{lstlisting}

This is a common pattern in JavaScript where function are passed in as call-back function when some unit of work has executed. We see more of that when we discuss the Node.js asynchronous model in section n.

One little function caveat. It is not required to pass value for every argument of the function. If not given, that argument will be \texttt{undefined}.

\begin{lstlisting}
> const foo = function(a, b, c) {
    console.log(a, b, c);
};
undefined
> foo(1);
1 undefined undefined
\end{lstlisting}

Where JavaScript functions accept less parameters then the ones that are defined, it would be strange if you couldn't add more then are defined in the argument list, right?\footnote{This makes no sense!}. Well, JavaScript has you covered. It is a good practice to validate that the arguments in functions have expected values. You have access to all the arguments passed into a given function with through \texttt{arguments} constant within the function. This features allows you to create functions with infinite\footnote{Bound to memory of course} arguments. Here is a simple example of how the \texttt{arguments} can be used to create a sum function.

\begin{lstlisting}
'use strict';

const mysum = function() {
  let sum = 0;
  for (let i of arguments) {
    sum += i;
  }
  return sum;
};

console.log(mysum(5, 2, 3));
\end{lstlisting}

\subsection{Fat Arrow functions}
In ES6 new type of function was introduced, namely the Fat Arrow Functions\footnote{\url{https://developer.mozilla.org/en-US/docs/Web/JavaScript/Reference/Functions/Arrow_functions}}. Now let's look at the syntactic difference between them and then the semantic difference.

Arrow functions are always anonymous functions and there is no special keyword to define them. This is the syntax for Arrow Functions

\begin{lstlisting}
(argument1, .. argumentn) => {statement1, .. statementn}
\end{lstlisting}

If there is only one argument, then the parentheses can be omitted. If there is only one statement the braces can be omitted.

\begin{lstlisting}
const add = (x, y) => x+y;
\end{lstlisting}

If the body of the function is a single expression, then the value of that statement is automatically returned. If the body has more than one statement, you must explicitly return a value with the `return` keyword.

This style of functions is shorter and is syntactically more beautiful, especially when it comes to call-back function as following example indicates.

\begin{lstlisting}
app.get('/news', function(req, res) {
    res.send('Hello world');
});


app.get('/news', (req, res) => res.send('Hello world'));    
\end{lstlisting}

Arrow functions are not a replacement for the old-style functions where they are semantically different. The difference lays in the scoping value of object `this`. When using Fat Arrow functions lexically bind the \texttt{this} value, that is to the enclosing context where the old style functions behave differently on that matter.

In this course this will not be of any problems. But we encourage students to look into the difference between those types of function and their handling of scopes.
